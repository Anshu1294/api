\chapter{Filters and Handlers}
\label{filters_and_handlers}

Filters and handlers can be registered for execution at well-defined extension points in a \jaxrs\ implementation. They are used to extend an implementation in order to provide capabilities such as logging, security, entity compression, etc. 

\section{Introduction}
\label{introduction}
Handlers wrap around a method invocation at a specific extension point. Filters execute code at an extension point but without wrapping a method invocation. There are two extension points for filters: Pre and Post. There are two extension points for handlers: ReadFrom and WriteTo.  For each of these extension points, there is a corresponding interface:

\begin{listing}{1}
public interface RequestFilter {
    FilterAction preFilter(FilterContext ctx) throws IOException;
}
public interface ResponseFilter {
    FilterAction postFilter(FilterContext ctx) throws IOException;
}
\end{listing}

\begin{listing}{1}
public interface ReadFromHandler<T> {
    T readFrom(ReadFromHandlerContext<T> context) throws IOException;
}
public interface WriteToHandler<T> {
    void writeTo(WriteToHandlerContext<T> context) throws IOException;
}
\end{listing}

A filter class is a class that implements \RequestFilter\ or \ResponseFilter\ (or both) and is annotated by \Provider. A handler class is a class that implements \ReadFromHandler\ or \WriteToHandler\ (or both) and is annotated by \Provider. 

In the Client API, filters implementing \RequestFilter\ MUST be executed \emph{before} the HTTP invocation and \emph{before} all handlers implementing \WriteToHandler; filters implementing \ResponseFilter\ MUST be executed \emph{after} the HTTP invocation and \emph{before} all handlers implementing \ReadFromHandler. 
In the Server API, filters implementing \RequestFilter\ MUST be executed \emph{before} the resource method is called and \emph{before} all handlers implementing \ReadFromHandler; filters implementing \ResponseFilter\ MUST be executed \emph{after} the HTTP invocation and \emph{before} all handlers implementing \WriteToHandler. 

\begin{ednote}
If filters can read or write entities, the rules above would be violated. For example, a server request filter could call \code{ctx.getRequest().getEntity(Customer.class)} forcing the ReadFrom handlers to be called. 
\end{ednote}

A handler implementing \ReadFromHandler\ wraps around calls to \code{MessageBodyReader.readFrom}. A handler implementing \WriteToHandler\ wraps around calls to \code{MessageBodyWrite.writeTo}. \jaxrs\ implementations are REQUIRED to call registered handlers when mapping representations to Java types and vice versa. See Section \ref{entity_providers} for more information on entity providers.

\section{Filters}
\label{filters}

As stated above, a filter implements interface \RequestFilter\ or \ResponseFilter\ or both. Multiple filters are grouped in \emph{filter chains}. Filters in a chain are sorted based on their priorities (see Section \ref{priorities}) and are executed in order. 

A call to a filter's \code{preFilter} or \code{postFilter} methods returns a \code{FilterAction} which is an enumeration of two values: \code{STOP} and \code{NEXT}. If a filter returns \code{NEXT}, implementations are REQUIRED to proceed with the rest of the filter chain; if a filter returns \code{STOP}, implementations are REQUIRED to abort execution of the filter chain. 

The following example shows an implementation of a logging filter: each method simply logs the message and returns \code{NEXT} to continue with the remainder of the filter chain.

\begin{listing}{1}
@Provider
public class LoggingFilter implements RequestFilter, ResponseFilter {

    @Override
    public FilterAction preFilter(FilterContext ctx) throws IOException {
        logRequest(ctx.getRequest());
        return FilterAction.NEXT;
    }

    @Override
    public FilterAction postFilter(FilterContext ctx) throws IOException {
        logResponse(ctx.getResponse());
        return FilterAction.NEXT;
    }
    ...
}
\end{listing}

A \FilterContext\ provides access to the request, the response (if available) as well as the ability to create and set new responses. Once the execution of a filter chain is completed, either by reaching the end of the chain or due to a filter returning \code{STOP}, \jaxrs\ implementations MUST get the response returned by \code{FilterContext.getResponse}. If this method returns null, normal execution is resumed; otherwise, the response returned is used and the HTTP invocation (Client API) or the resource method invocation (Server API) is omitted.  See Appendix \ref{extension_points} for more information.

\section{Handlers}

A handler implements interface \ReadFromHandler\ or \WriteToHandler\ or both. Multiple handlers are grouped in \emph{handler chains}. Handlers in a chain are sorted based on their priorities (see Section \ref{priorities}) and are executed in order. 

Handlers wrap calls to the methods \code{MessageBodyReader.readFrom} or \code{MessageBodyWrite.writeTo}. Handlers SHOULD explicitly call the context method \code{proceed}  to continue the execution of the chain. Because of their wrapping nature, failure to call this method will prevent execution of the wrapped method in the corresponding message body reader or message body writer.

The following example shows an implementation of a GZIP handler that provides deflate and inflate capabilities~\footnote{This class is not intended to be a complete implementation of a GZIP handler.}

\begin{listing}{1}
@Provider
public class GzipHandler implements ReadFromHandler, WriteToHandler {

    @Override
    public Object readFrom(ReadFromHandlerContext ctx) throws IOException {
        InputStream old = ctx.getInputStream();
        ctx.setInputStream(new GZIPInputStream(old));
        try {
            return ctx.proceed();
        } finally {
            ctx.setInputStream(old);
        }
    }

    @Override
    public void writeTo(WriteToHandlerContext ctx) throws IOException {
        OutputStream old = ctx.getOutputStream();
        GZIPOutputStream gzipOutputStream = new GZIPOutputStream(old);
        ctx.setOutputStream(gzipOutputStream);
        try {
            ctx.proceed();
        } finally {
            gzipOutputStream.finish();
            ctx.setOutputStream(old);
        }
    }
    ...
}
\end{listing}

The context types, \ReadFromHandlerContext\ and \WriteToHandlerContext, provide read and write access to the parameters of the corresponding wrapped methods. In the example shown above, the input and output streams are wrapped and updated in the context object before proceeding. \jaxrs\ implementations MUST use the last parameter values set in the context object when calling the wrapped methods \code{MessageBodyReader.readFrom} and \code{MessageBodyWrite.writeTo}.

\section{Lifecycle}

By default, just like all the other providers, a single instance of each filter or handler is instantiated for each \jaxrs\ application. First the constructor is called, then any requested dependencies are injected, then the appropriate methods are called (simultaneously) as needed. Implementations MAY offer alternative lifecycle options beyond the default one. See Section \ref{lifecycle_and_environment} for additional information.

\section{Binding}

Binding is the process by which a handler or filter is associated with a resource class or method (Server API) or an invocation (Client API). The forms of binding presented in the next sections are only supported as part of the Server API. See Section \ref{binding_in_client_api} for binding in the Client API.

\subsection{Name Binding}
\label{Name_Binding}

A handler or filter can be associated with a resource class or method by declaring a new \emph{binding} annotation \`{a} la CDI~\cite{jsr299}. These annotations are declared using the \jaxrs\ meta-annotation \NameBinding\ and are used to decorate both the filter (or handler) and the resource method (or class). For example, the \code{LoggingFilter} defined in Section \ref{filters} can be bound to the method \code{hello} in \code{MyResourceClass} as follows:

\begin{listing}{1}
@Provider
@Logged
public class LoggingFilter implements RequestFilter, ResponseFilter {
    ...
}
\end{listing}

\begin{listing}{1}
@Path("/")
public class MyResourceClass {
    @Logged
    @GET
    @Produces("text/plain")
    @Path("{name}")
    public String hello(@PathParam("name") String name) {
        return "Hello " + name;
    }
}
\end{listing}

According to the semantics of \code{LoggingFilter}, the request will be logged before the \code{hello} method is called and the response will be logged after it returns. The declaration of the \code{@Logged} annotation is shown next.

\begin{listing}{1}
@NameBinding
@Target({ ElementType.TYPE, ElementType.METHOD })
@Retention(value = RetentionPolicy.RUNTIME)
public @interface Logged { }
\end{listing}

Binding annotations that decorate resource classes apply to all the resource methods defined in them. A filter or handler class can be decorated with multiple binding annotations. Similarly, a resource method can be decorated with multiple binding annotations.  Each binding annotation instance in a resource method denotes a set of filters and handlers whose class definitions are decorated (possibly among others) with that annotation. The final set of (static) filters and handlers is the union of all these sets \footnote{By definition of {\em set}, any duplicate filters or handlers in each individual set or the final set are eliminated.}; this set must be sorted based on priorities as explained in Section \ref{priorities}. 

\subsection{Global Binding}

Name binding is a form of {\em local} binding in which filters or handlers are bound to specific methods or classes. Occasionally, it is useful to declare a handler or filter as being bound to all the resource classes in an application. This can be accomplished by using the \GlobalBinding\ annotation directly on the handler or filter class without the need to declare a new binding annotation. For example, the \code{LoggingFilter} defined in Section \ref{filters} can be defined as follows:

\begin{listing}{1}
@Provider
@GlobalBinding
public class LoggingFilter implements RequestFilter, ResponseFilter {
    ...
}
\end{listing}

If this filter is registered as part of an application, requests and responses will be logged for all resource methods. 

A filter or handler whose class is decorated with \GlobalBinding\ cannot be associated with a resource class or method using name binding. Implementations are REQUIRED to report an error if a filter or handler is annotated with both \GlobalBinding\ and any other annotation derived from \NameBinding.

\subsection{Dynamic Binding}

The annotation-based forms of binding presented thus far are {\em static}.  By having a filter or handler implement the \DynamicBinding\ interface, {\em dynamic} variants of name binding and global binding are possible. In these cases, binding is determined based on (i) the filter or handler being associated to a resource method globally or by name and (ii) the value returned by the \code{isBound} method in \DynamicBinding. 

The following example defines a global \code{LoggingFilter} that is bound dynamically to all resource methods in \code{MyResourceClass} that are annotated by \code{@GET}.

\begin{listing}{1}
@Provider
@GlobalBinding
public class LoggingFilter implements RequestFilter, DynamicBinding {

    public boolean isBound(Class<?> type, Method method) {
        return type.equals(MyResourceClass.class) 
            && method.isAnnotationPresent(GET.class);    
    }
    ...
}
\end{listing}



\subsection{Binding in Client API}
\label{binding_in_client_api}

Binding in the Client API is accomplished via API calls instead of annotations. More specifically, filters and handlers are registered like any other provider by calling \code{Configurable.register} methods. \Client, \Invocation, \InvocationBuilder\ and \Target\ all implement the \Configurable\ interface. See Sections \ref{configurable_types} for more information.

\section{Priorities}
\label{priorities}

The order in which handlers or filters are executed as part of their corresponding chains is controlled by the \BindingPriority\ annotation.
Priorities are represented by integer numbers: the \emph{lower} the number, the \emph{higher} the priority. The default priority for all handlers and filters---when an instance of \BindingPriority\ is absent or is present but without any value specified---is \code{BindingPriority.USER}. 

The \code{BindingPriority} class defines additional built-in priorities for security, header decorators, decoders and encoders. For example, the priority of authentication filter can be set as follows:

\begin{listing}{1}
@Provider
@Authenticated
@BindingPriority(BindingPriority.SECURITY)
public class AuthenticationFilter implements RequestFilter {
    ...
}
\end{listing}

Note that even though, as explained in Section \ref{binding_in_client_api}, annotations are not used for binding in the Client API, they are still used to define priorities. Therefore, if a priority other than the default is required, the \BindingPriority\ annotation must be used for a filter or handler registered using the Client API. 

Implementations are REQUIRED to sort filters and handlers according to \code{BindingPriority.value()}, using \code{BindingPriority.USER} as the default if the annotation is absent. The order in which filters or handlers that belong to the same priority class are executed is implementation dependent.



