\chapter{Providers}
\label{providers}

The \jaxrs\ runtime is extended using application-supplied provider classes. A provider is a class that implements one or more \jaxrs\ interfaces introduced in this specification and that may be annotated with \Provider\ for automatic discovery. This chapter introduces some of the basic \jaxrs\ providers; other providers are introduced in Chapter \ref{client_api} and Chapter \ref{filters_and_interceptors}.

\section{Lifecycle and Environment}
\label{lifecycle_and_environment}

By default a single instance of each provider class is instantiated for each \jaxrs\ application. First the constructor (see Section \ref{provider_class_constructor}) is called, then any requested dependencies are injected (see Chapter \ref{context}), then the appropriate provider methods may be called multiple times (simultaneously), and finally the object is made available for garbage collection. Section \ref{providercontext} describes how a provider obtains access to other providers via dependency injection.

An implementation MAY offer other provider lifecycles, mechanisms for specifying these are outside the scope of this specification. E.g. an implementation based on an inversion-of-control framework may support all of the lifecycle options provided by that framework.

\subsection{Automatic Discovery}
\label{automatic_discovery}

The annotation \Provider\ is used by a \jaxrs\ runtime to automatically discover provider classes via mechanisms such as class scanning. A \jaxrs\ implementation that supports automatic discovery of classes MUST process only those classes that are annotated with \Provider.

\subsection{Constructors}
\label{provider_class_constructor}

Provider classes that are instantiated by the \jaxrs\ runtime and MUST have a public constructor for which the \jaxrs\ runtime can provide all parameter values. Note that a zero argument constructor is permissible under this rule.

A public constructor MAY include parameters annotated with \Context ---Chapter \ref{context} defines the parameter types permitted for this annotation. Since providers may be created outside the scope of a particular request, only deployment-specific properties may be available from injected interfaces at construction time; request-specific properties are available when a provider method is called. If more than one public constructor can be used then an implementation MUST use the one with the most parameters. Choosing amongst constructors with the same number of parameters is implementation specific, implementations SHOULD generate a warning about such ambiguity.

\section{Entity Providers}
\label{entity_providers}

Entity providers supply mapping services between representations and their associated Java types. Entity providers come in two flavors: \MsgRead\ and \MsgWrite\ described below. 

\subsection{Message Body Reader}
\label{message_body_reader}

The \MsgRead\ interface defines the contract between the \jaxrs\ runtime and components that provide mapping services from representations to a corresponding Java type. A class wishing to provide such a service implements the \MsgRead\ interface and may be annotated with \Provider\ for automatic discovery.

The following describes the logical\footnote{Implementations are free to optimize their processing provided the results are equivalent to those that would be obtained if these steps are followed.} steps taken by a \jaxrs\ implementation when mapping a request entity body to a Java method parameter:

\begin{enumerate}
\item Obtain the media type of the request. If the request does not contain a \code{Content-Type} header then use \code{application/octet-stream}.
\item Identify the Java type of the parameter whose value will be mapped from the entity body. Section \ref{mapping_requests_to_java_methods} describes how the Java method is chosen.
\item Select the set of \MsgRead\ classes that support the media type of the request, see Section \ref{declaring_provider_capabilities}.
\item\label{findreader} Iterate through the selected \MsgRead\ classes and, utilizing the \code{isReadable} method of each, choose a \MsgRead\ provider that supports the desired Java type.
\item If step \ref{findreader} locates a suitable \MsgRead\ then use its \code{readFrom} method to map the entity body to the desired Java type.
\item Else generate a \WebAppExc\ that contains an unsupported media type response (HTTP 415 status) and no entity. The exception MUST be processed as described in Section \ref{method_exc}.
\end{enumerate}

A \MsgRead\code{.readFrom} method MAY throw \WebAppExc. See Section \ref{exceptions_providers} for more information on exception handling.

\subsection{Message Body Writer}
\label{message_body_writer}

The \MsgWrite\ interface defines the contract between the \jaxrs\ runtime and components that provide mapping services from a Java type to a representation. A class wishing to provide such a service implements the \MsgWrite\ interface and may be annotated with \Provider\ for automatic discovery.

The following describes the logical steps taken by a \jaxrs\ implementation when mapping a return value to a response entity body:

\begin{enumerate}
\item Obtain the object that will be mapped to the response entity body. For a return type of \Response\ or subclasses, the object is the value of the \code{entity} property, for other return types it is the returned object.
\item Determine the media type of the response, see Section \ref{determine_response_type}.
\item Select the set of \MsgWrite\ providers that support (see Section \ref{declaring_provider_capabilities}) the object and media type of the response entity body.
\item\label{item_sort} Sort the selected \MsgWrite\ providers with a primary key of generic type where providers whose generic type is the nearest superclass of the object class are sorted first and a secondary key of media type (see Section \ref{declaring_provider_capabilities}).
\item\label{findwriter} Iterate through the sorted \MsgWrite\ providers and, utilizing the \code{isWriteable} method of each, choose an \MsgWrite\ that supports the object that will be mapped to the entity body.
\item If step \ref{findwriter} locates a suitable \MsgWrite\ then use its \code{writeTo} method to map the object to the entity body. 
\item Else generate an \code{InternalServerErrorException}, a subclass of \WebAppExc\ with its status set to 500, and no entity. The exception MUST be processed as described in Section \ref{method_exc}.
\end{enumerate}

Experience gained in the field has resulted in the reversal of the sorting keys in step
\ref{item_sort} in this specification. This represents a backward incompatible change
with respect to \jaxrs\ 1.X. Implementations of this specification are REQUIRED to
provide a backward compatible flag for those applications that rely on the previous
ordering. The mechanism defined to enable this flag is implementation dependent.

A \MsgWrite\code{.write} method MAY throw \WebAppExc. See Section \ref{exceptions_providers} for more information on exception handling.

\subsection{Declaring Media Type Capabilities}
\label{declaring_provider_capabilities}

Message body readers and writers MAY restrict the media types they support using the \Consumes\ and \Produces\ annotations respectively. The absence of these annotations is equivalent to their inclusion with media type (\lq\lq*/*\rq\rq), i.e. absence implies that any media type is supported. An implementation MUST NOT use an entity provider for a media type that is not supported by that provider.

When choosing an entity provider an implementation sorts the available providers according to the media types they declare support for. Sorting of media types follows the general rule: x/y $<$ x/* $<$ */*, i.e. a provider that explicitly lists a media types is sorted before a provider that lists */*.

\subsection{Standard Entity Providers}
\label{standard_entity_providers}
 
An implementation MUST include pre-packaged \MsgRead\ and \MsgWrite\ implementations for the following Java and media type combinations:

\begin{description}
\item[\code{byte[]}] All media types (\code{*/*}).
\item[\code{java.lang.String}] All media types (\code{*/*}).
\item[\code{java.io.InputStream}] All media types (\code{*/*}).
\item[\code{java.io.Reader}] All media types (\code{*/*}).
\item[\code{java.io.File}] All media types (\code{*/*}).
\item[\code{javax.activation.DataSource}] All media types (\code{*/*}).
\item[\code{javax.xml.transform.Source}] XML types (\code{text/xml}, \code{application\-/\-xml} and \code{application\-/\-*+xml}).
\item[\code{javax.xml.bind.JAXBElement} and application-supplied JAXB classes] XML media types (\code{text\-/\-xml}, \code{application/xml} and \code{application/*+xml}).
\item[\code{MultivaluedMap<String,String>}] Form content (\code{application/x-www-form-urlencoded}).
\item[\code{StreamingOutput}] All media types (\code{*/*}), \MsgWrite\ only.
\item[\code{java.lang.Boolean}, \code{java.lang.Character}, \code{java.lang.Number}] Only for \code{text/plain}. Corresponding primitive types supported via boxing/unboxing conversion.
\end{description}

When reading zero-length request entities, all pre-packaged \MsgRead\ implementations except the JAXB-related one MUST create a corresponding Java object that represents zero-length data; they MUST NOT return null. The pre-packaged JAXB \MsgRead\ implementation MUST throw a \WebAppExc\ with a client error response (HTTP 400) for zero-length request entities. Moreover, all pre-packaged \MsgRead\ implementations SHOULD throw a \WebAppExc\ with a client error response (HTTP 400) if any other error is encountered while
reading a request entity.

The implementation-supplied entity provider(s) for \code{javax\-.xml\-.bind\-.JAXBElement} and application-supplied JAXB classes MUST use \code{JAXBContext} instances provided by application-supplied context resolvers, see Section \ref{contextprovider}. If an application does not supply a \code{JAXBContext} for a particular type, the implementation-supplied entity provider MUST use its own default context instead.

When writing responses, implementations SHOULD respect application-supplied character set metadata and SHOULD use UTF-8 if a character set is not specified by the application or if the application specifies a character set that is unsupported.

An implementation MUST support application-provided entity providers and MUST use those in preference to its own pre-packaged providers when either could handle the same request. More precisely, step \ref{findreader} in Section \ref{message_body_reader} and step \ref{findwriter} in Section \ref{message_body_writer} MUST prefer application-provided over pre-packaged entity providers.

\subsection{Transfer Encoding}
\label{transfer_encoding}

Transfer encoding for inbound data is handled by a component of the container or the \jaxrs\ runtime. \MsgRead\ providers always operate on the decoded HTTP entity body rather than directly on the HTTP message body.

A JAX-RS runtime or container MAY transfer encode outbound data or this MAY be done by application code.

\subsection{Content Encoding}

Content encoding is the responsibility of the application. Application-supplied entity providers MAY perform such encoding and manipulate the HTTP headers accordingly.

\section{Context Providers}
\label{contextprovider}

Context providers supply context to resource classes and other providers. A context provider class implements the \code{ContextResolver<T>} interface and may be annotated with \Provider\ for automatic discovery. E.g., an application wishing to provide a customized \code{JAXBContext} to the default JAXB entity providers would supply a class implementing \code{ContextResolver<JAXBContext>}.

Context providers MAY return \code{null} from the \code{getContext} method if they do not wish to provide their context for a particular Java type. E.g. a JAXB context provider may wish to only provide the context for certain JAXB classes. Context providers MAY also manage multiple contexts of the same type keyed to different Java types.

\subsection{Declaring Media Type Capabilities}
\label{context_media_type}

Context provider implementations MAY restrict the media types they support using the \Produces\ annotation. The absence of this annotation is equivalent to its inclusion with media type (\lq\lq*/*\rq\rq), i.e. absence implies that any media type is supported.

When choosing a context provider an implementation sorts the available providers according to the media types they declare support for. Sorting of media types follows the general rule: x/y $<$ x/* $<$ */*, i.e. a provider that explicitly lists a media type is sorted before a provider that lists */*.

\section{Exception Mapping Providers}
\label{exceptionmapper}

Exception mapping providers map a checked or runtime exception to an instance of \Response. An exception mapping provider implements the \code{ExceptionMapper<T>} interface and may be annotated with \Provider\ for automatic discovery. When choosing an exception mapping provider to map an exception, an implementation MUST use the provider whose generic type is the nearest superclass of the exception.

When a resource class or provider method throws an exception for which there is an exception mapping provider, the matching provider is used to obtain a \Response\ instance. The resulting \Response\ is processed as if a web resource method had returned the \Response, see Section \ref{resource_method_return}. In particular, a mapped \Response\ MUST be processed using the ContainerResponse filter chain defined in  Chapter~\ref{filters_and_interceptors}.

To avoid a potentially infinite loop, a single exception mapper must be used during the processing of a request and its corresponding response. \jaxrs\ implementations MUST NOT attempt to map exceptions thrown while processing a response previously mapped from an exception. Instead, this exception MUST be processed as described in steps \ref{runtimeexc} and \ref{checkedexc} in Section \ref{method_exc}.

Note that exception mapping providers are {\em not} supported as part of the Client API.

\section{Exceptions}
\label{exceptions_providers}

When a provider method throws an exception, the \jaxrs\ runtime will attempt to map the exception to a suitable HTTP response in the same way as described for methods and locators in Section \ref{method_exc}. If the exception is thrown
while generating a response, \jaxrs\ implementations are required to map the exception {\em only when} the response has not been committed yet.

As explained in Section \ref{exceptionmapper}, an application can supply exception mapping providers to customize this mapping, but these exception mappers will be ignored during the processing of a {\em previously mapped} response to avoid entering a potentially infinite loop. For example, suppose a method in a message body reader throws an exception that is mapped to a response via an exception mapping provider; if the message body writer throws an exception while trying to write the mapped response, \jaxrs\ implementations will not attempt to map the exception again.






