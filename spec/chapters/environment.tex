\chapter{Environment}
\label{environment}

The container-managed resources available to a \jaxrs\ root resource class or provider depend on the environment in which it is deployed. Section \ref{contexttypes} describes the types of context available regardless of container. The following sections describe the additional container-managed resources available to a \jaxrs\ root resource class or provider deployed in a variety of environments.

\section{Servlet Container}
\label{servlet_container}

The \Context\ annotation can be used to indicate a dependency on a Servlet-defined resource. A Servlet-based implementation MUST support injection of the following Servlet-defined types: \code{Servlet\-Config}, \code{Servlet\-Context}, \code{Http\-Servlet\-Request} and \code{Http\-Servlet\-Response}.

An injected \code{Http\-Servlet\-Request} allows a resource method to stream the contents of a request entity. If the resource method has a parameter whose value is derived from the request entity then the stream will have already been consumed and an attempt to access it MAY result in an exception.

An injected \code{Http\-Servlet\-Response} allows a resource method to commit the HTTP response prior to returning. An implementation MUST check the committed status and only process the return value if the response is not yet committed.

Servlet filters may trigger consumption of a request body by accessing request parameters. In a servlet container the @FormParam annotation and the standard entity provider for \code{application\-/x-\-www-\-form-\-urlencoded} MUST obtain their values from the servlet request parameters if the request body has already been consumed. Servlet APIs do not differentiate between parameters in the URI and body of a request so URI-based query parameters may be included in the entity parameter.

\section{Java EE Container}
\label{javaee}

This section describes the additional requirements that apply to a \jaxrs\ implementation when combined in a product that supports the following specifications. 

\subsection{Servlets}
\label{servlets}

In a product that also supports the Servlet specification, implementations MUST support \jaxrs\ applications that are packaged as a Web application.
See Section \ref{servlet} for more information Web application packaging.

It is RECOMMENDED for a \jaxrs\ implementation to provide asynchronous processing support, as defined in Chapter \ref{asynchronous_processing}, by enabling asynchronous processing (i.e., \code{asyncSupported=true}) in the underlying Servlet 3 container. It is OPTIONAL for a \jaxrs\ implementation to support asynchronous processing when running on a Servlet container whose version is prior to 3.

As explained in Section \ref{servlet_container}, injection of Servlet-defined types is possible using the \Context\ annotation.

\subsection{Managed Beans}
\label{managed_beans}

In a product that supports Managed Beans, implementations MUST support the use of Managed Beans as root resource classes, providers and \code{Application} subclasses. 

For example, a bean that uses a managed-bean interceptor can be defined as a \jaxrs\ resource as follows:

\begin{listing}{1}
@ManagedBean
@Path("/managedbean")
public class ManagedBeanResource {

    public static class MyInterceptor {
        @AroundInvoke
        public String around(InvocationContext ctx) throws Exception {
            System.out.println("around() called");
            return (String) ctx.proceed();
        }
    }

    @GET
    @Produces("text/plain")
    @Interceptors(MyInterceptor.class)
    public String getIt() {
        return "Hi managedbean!";
    }
}
\end{listing}

The example above uses a managed-bean interceptor to intercept calls to the resource method \code{getIt}. See Section \ref{additional_reqs} for additional requirements on Managed Beans.

\subsection{Context and Dependency Injection (CDI)}
\label{cdi}
In a product that supports CDI, implementations MUST support the use of CDI-style Beans as root resource classes, providers and \code{Application} subclasses. Providers and \code{Application} subclasses MUST be singletons or use application scope. 

For example, assuming CDI is enabled via the inclusion of a \code{beans.xml} file, a CDI-style bean that can be defined as a \jaxrs\ resource as follows:

\begin{listing}{1}
@Path("/cdibean")
public class CdiBeanResource {

	@Inject MyOtherCdiBean bean;		// CDI injected bean

    @GET
    @Produces("text/plain")
    public String getIt() {
        return bean.getIt();
    }
}
\end{listing}

The example above takes advantage of the type-safe dependency injection provided in CDI by using another bean, of type \code{MyOtherCdiBean}, in order to return a resource representation. See Section \ref{additional_reqs} for additional requirements on CDI-style Beans.

\subsection{Enterprise Java Beans (EJBs)}
\label{ejbs}

In a product that supports EJBs, an implementation MUST support the use of stateless and singleton session beans as root resource classes, providers and \code{Application} subclasses. 
\jaxrs\ annotations can be applied to methods in an EJB's local interface or directly to methods in a no-interface EJB. Resource class annotations (like \Path) MUST be applied to an EJB's class directly following the annotation inheritance rules defined in Section \ref{annotationinheritance}.

For example, a stateless EJB that implements a local interface can be defined as a \jaxrs\ resource class as follows:

\begin{listing}{1}
@Local
public interface LocalEjb {

    @GET
    @Produces("text/plain")
    public String getIt();
}

@Stateless
@Path("/stateless")
public class StatelessEjbResource implements LocalEjb {

    @Override
    public String getIt() {
        return "Hi stateless!";
    }
}
\end{listing}

\jaxrs\ implementations are REQUIRED to discover EJBs by inspecting annotations on classes and local interfaces; they are not REQUIRED to read EJB deployment descriptors (ejb-jar.xml). Therefore, any information in an EJB deployment descriptor for the purpose of overriding EJB annotations or providing additional meta-data will likely result in a non-portable \jaxrs\ application. 

If an \code{Exception\-Mapper} for a \code{EJBException} or subclass is not included with an application then exceptions thrown by an EJB resource class or provider method MUST be unwrapped and processed as described in Section \ref{method_exc}. 

See Section \ref{async_ejbs} for more information on asynchronous EJB methods and Section \ref{additional_reqs} for additional requirements on EJBs.

\subsection{Additional Requirements}
\label{additional_reqs}

The following additional requirements apply when using Managed Beans, CDI-style Beans or EJBs as resource classes, providers or \code{Application} subclasses:

\begin{itemize}
\item Field and property injection of \jaxrs\ resources MUST be performed prior to the container invoking any \code{@PostConstruct} annotated method.
\item Support for constructor injection of \jaxrs\ resources is OPTIONAL. Portable applications MUST instead use fields or bean properties in conjunction with a \code{@PostConstruct} annotated method. Implementations SHOULD warn users about use of non-portable constructor injection.
\item Implementations MUST NOT require use of \code{@Inject} or \code{@Resource} to trigger injection of \jaxrs\ annotated fields or properties. Implementations MAY support such usage but SHOULD warn users about non-portability.
\end{itemize}

\section{Other}

Other container technologies MAY specify their own set of injectable resources but MUST, at a minimum, support access to the types of context listed in Section \ref{contexttypes}.
