\chapter{Client API}
\label{client_api}

The Client API is used to access Web resources. It provides a higher-level API than \code{HttpURLConnection} as well as integration with JAX-RS providers. Unless otherwise stated, types presented in this chapter live in the \code{javax.ws.rs.client} package.

\section{Bootstrapping a Client Instance}

An instance of \Client\ is required to access a Web resource using the Client API. The default instance of \Client\ can be obtained by calling \code{newClient} on \ClientFactory. \Client\ instances can be configured by calling the \code{configuration} method; the object returned, of type \code{Configuration}, provides access to providers, properties and features:

\begin{listing}{1}
// Default instance of client
Client client = ClientFactory.newClient();

// Additional configuration of default client
client.configuration()
    .setProperty("MyProperty", "MyValue")
    .register(MyProvider.class)
    .enable(MyFeature.class);
\end{listing}

See Chapter \ref{providers} for more information on providers. Properties are simply name-value pairs where the value is an arbitrary object.  Features must implement the \code{Feature} interface by providing concrete implementations of the \code{enable} and \code{disable} methods; they are useful for grouping sets of properties and providers (or other features) that are logically related and must be enabled and disabled as a unit.

An reference to a \code{ClientBuilderFactory} class can be provided in order to obtain customized instances of \code{Client} which may provide functionality beyond what it is described in this document.  

\begin{listing}{1}
// Custom client using different builder factory class
MyClient myClient = 
    ClientFactory.newClientBy(MyClientBuilderFactory.class).build();
myClient.enableCaching(true);
\end{listing}

Note that in this example the \Client\ instance returned by the factory is of type \code{MyClient}. The method \code{enableCaching} is not defined in the default client shown in the previous example.

\section{Resource Access}
\label{resource_access}

A Web resource can be accessed using a fluent API in which methods invocations are chained to build and ultimately submit an HTTP request. The following example gets a \code{text/plain} representation of the resource identified by \code{http://example.org/hello}:

\begin{listing}{1}
Client client = ClientFactory.newClient();
Response res = client.target("http://example.org/hello")
    .request("text/plain").get();
\end{listing}

Conceptually, the steps required to submit a request are the following: (i) obtain an instance of \Client\ (ii) create a \Target\ (iii) create a request from the \Target\ and (iv) submit a request or get a prepared \Invocation\ for later submission. See Section \ref{invocations} for more information on using \Invocation.

Method chaining is not limited to the example shown above. A request can be further specified by setting headers, cookies, query parameters, etc. For example:

\begin{listing}{1}
Response res = client.target("http://example.org/hello")
    .queryParam("MyParam","...")
    .request("text/plain")
    .header("MyHeader", "...")
    .get();
\end{listing}

See the Javadoc for the classes in the \code{javax.ws.rs.client} package for more information.

\section{Targets}

The benefits of using a \code{Target} become apparent when building complex URIs, for example by extending base URIs with additional path segments or using URI templates. The following example highlights these features:

\begin{listing}{1}
Target base = client.target("http://example.org/");
Target hello = base.path("hello").path("{whom}");   
Response res = hello.pathParam("whom", "world").request("...").get();
\end{listing}

Note the use of the URI template parameter~\code{\{whom\}}. The example above gets a representation for the resource identified by \code{http://example.org/hello/world}.

\Target\ instances are {\em immutable} with respect to their URI (or URI template): methods for specifying additional path segments and parameters return a new instance of \Target. However, \Target\ instances are {\em mutable} with respect to their configuration. Thus, configuring a \Target\ does not create new instances.

\begin{listing}{1}
// Create Target instance base
Target base = client.target("http://example.org/");
// Create new Target instance hello and configure
Target hello = base.path("hello");
hello.configuration().register(MyProvider.class);
\end{listing}

In this example, two instances of \Target\ are created. The instance \code{hello} inherits the configuration from \code{base} and it is further configured by adding \code{MyProvider.class}. Note that changes to \code{hello}'s configuration do not affect \code{base}, i.e.~configuration inheritance requires performing a deep copy of the configuration. See Section \ref{configurable_types} for additional information.

\section{Typed Entities}

The response to a request is not limited to be of type \Response. The following example upgrades the status of customer number 123 to ``gold status'' by first obtaining an entity of type \code{Customer} and then posting that entity to a different URI:

\begin{listing}{1}
Customer c = client.target("http://examples.org/customers/123")
    .request("application/xml").get(Customer.class);
String newId = client.target("http://examples.org/gold-customers/")
    .request().post(xml(c), String.class);
\end{listing}

Note the use of the {\em variant} \code{xml()} in the call to \code{post}. The class \code{javax.ws.rs.client.Entity} defines variants for the most popular media types used in JAX-RS applications.

In the example above, just like in the Server API, \jaxrs\ implementations are REQUIRED to use entity providers to map a representation of type \code{"application/xml"} to an instance of \code{Customer} and vice versa. See Section \ref{standard_entity_providers} for a list of entity providers that MUST be supported by all \jaxrs\ implementations.

\section{Invocations}
\label{invocations}

An invocation is a request that has been prepared and is ready for execution. Invocations provide a {\em generic interface} that enables a separation of concerns between the creator and the submitter. In particular, the submitter does not need to know how the invocation was prepared, but only how it should be executed: namely, synchronously or asynchronously.

 Let us consider the following example\footnote{The Collections class in this example is arbitrary and does not correspond to any specific implementation. There are a number of Java collection libraries available that provide this type of functionality.}:

\begin{listing}{1}
// Executed by the creator
Invocation inv1 = client.target("http://examples.org/atm/balance")
    .queryParam("card", "111122223333").queryParam("pin", "9876")
    .request("text/plain").buildGet();
Invocation inv2 = client.target("http://examples.org/atm/withdrawal")
    .queryParam("card", "111122223333").queryParam("pin", "9876")
    .request().buildPost(text("50.0")));
Collection<Invocation> invs = Arrays.asList(inv1, inv2);

// Executed by the submitter
Collection<Response> ress =
    Collections.transform(invs,
        new F<Invocation, Response>() {
             public Response apply(Invocation inv) {
                 return inv.invoke(); } });
\end{listing}

In this example, two invocations are prepared and stored in a collection by the creator. The submitter then traverses the collection applying a transformation that maps an \Invocation\ to an \Response. The mapping calls \code{Invocation.invoke()} to execute the invocation synchronously; asynchronou execution is also supported by calling \code{Invocation.submit()}.

\section{Configurable Types}
\label{configurable_types}

The following Client API types are configurable: \Client, \Invocation, \InvocationBuilder\ and \Target. In all cases, the configuration can be accessed by calling the \code{configuration} method. This interface supports configuration of:

\begin{description}
\item [Features] Instances of classes that implement \code{Feature} and can be enabled or disabled in order to configure a \jaxrs\ implementation.
\item [Properties] Name-value pairs for additional configuration of features or other components of a \jaxrs\ implementation.
\item [Providers] Classes or instances of classes annotated with \Provider. A provider can be a message body reader, a filter, a context resolver, etc. See Chapter \ref{providers} for more information.
\end{description}

A configuration defined on an instance of any of the aforementioned types is inherited by other instances created from it. For example, an instance of \Target\ created from a \Client\ will inherit its configuration. However, any additional changes to the instance of \Target\ will not impact the \Client's configuration and vice versa. Therefore, once a configuration is inherited it is also detached (deep copied) from its parent configuration and changes to the parent and child configurations are not be visible to each other.

\subsection{Filters and Entity Interceptors}
\label{filters_interceptors_client}

As explained in Chapter \ref{filters_and_interceptors}, filters and interceptors are defined as JAX-RS providers. Therefore, they can be registered in any of the configurable types listed in the previous section. The following example shows how to register filters and interceptors on instances of \Client, \Target\ and \Invocation:

\begin{listing}{1}
// Create client and register logging filter
Client client = ClientFactory.newClient();
client.configuration().register(LoggingFilter.class);

// Executes logging filter from client and caching filter from target
Target t = client.target("http://examples.org/customers/123");
t.configuration().register(CachingFilter.class);
Customer c = t.request("application/xml").get(Customer.class);

// Executes logging filter from client and gzip interceptor from invocation
Invocation i = client.target("http://examples.org/gold-customers/")
    .request().buildPost(xml(c));
i.configuration().register(GzipInterceptor.class);
String newId = i.invoke(String.class);
\end{listing}

In this example, \code{LoggingFilter} is inherited by each instance of \Target\ created from \code{client}; the providers \code{CachingFilter} and \code{GzipInterceptor} are defined on a \Target\ and an \Invocation, respectively.








