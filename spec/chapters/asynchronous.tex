\chapter{Asynchronous Processing}
\label{asynchronous_processing}

This chapter describes the asynchronous processing capabilities in JAX-RS. Asynchronous processing is supported both in the Client API and in the Server API.

\section{Introduction}
\label{introduction_async}

Asynchronous processing is a technique that enables a better and more efficient use of processing threads. On the client side, a thread that issues a request may also be responsible for updating a UI component; if that thread is blocked waiting for a response, the user's perceived performance of the application will suffer. Similarly, on the server side, a thread that is processing a request should avoid blocking while waiting for an external event to complete so that other requests arriving to the server during that period of time can be attended\footnote{The maximum number of request threads is typically set by the administrator; if that upper bound is reached, subsequent requests will be rejected.}.

\section{Server API}
\label{server_api}

Synchronous processing requires a resource method to produce a response upon returning control back to the JAX-RS implementation. Asynchronous processing enables a resource method to inform the JAX-RS implementation that a response is not readily available upon return but will be produced at a future time. This can be accomplished by first {\em suspending} and later {\em resuming} the client connection on which the request was received. 

Let us illustrate these concepts via an example:

\begin{listing}{1}
@Path("/async/longRunning")
public class MyResource {
    
    @GET
    public void longRunningOp(@Suspended final AsyncResponse ar) {
        executor.submit(
            new Runnable() {
                public void run() {
                    executeLongRunningOp();
                    ar.resume("Hello async world!");
        } });
    } 
    ...
}
\end{listing}

A resource method that elects to produce a response asynchronously must inject as a method parameter an instance of the class \code{AsyncResponse} using the special annotation \Suspended. In the example above, the method \code{longRunningOp} is called upon receiving a \code{GET} request. Rather than producing a response immediately, this method forks a (non-request) thread to execute a long running operation and returns immediately. Once the execution of the long running operation is complete, the connection is resumed and the response returned by calling \code{resume} on the injected instance of \code{AsyncResponse}. 

For more information on executors, concurrency and thread management in a Java EE environment, the reader is referred to JSR 236 \cite{concurrencyee}.

\subsection{Timeouts and Callbacks}
\label{timeouts_and_callbacks}

A timeout value can be specified when suspending a connection to avoid waiting for a response indefinitely. The default unit is milliseconds, but any unit of type \code{java.util.concurrent.TimeUnit} can be used. The following example sets a timeout of 15 seconds and registers an instance of \code{TimeoutHandler} in case the timeout is reached before the connection is resumed.

\begin{listing}{1}
    @GET
    public void longRunningOp(@Suspended final AsyncResponse ar) {
        // Register handler and set timeout
        ar.setTimeoutHandler(new TimeoutHandler() {
            public void handleTimeout(AsyncResponse ar) {
                ar.resume(Response.status(SERVICE_UNAVAILABLE).entity(
                    "Operation timed out -- please try again").build());                    
                }
        });
        ar.setTimeout(15, SECONDS);
        
        // Execute long running operation in new thread
        executor.execute(
            new Runnable() {
                public void run() {
                    executeLongRunningOp();
                    ar.resume("Hello async world!");
        } });
    }
\end{listing}

\jaxrs\ implementations are REQUIRED to generate a \code{ServiceUnavailableException}, a subclass of \WebApplicationException\ with its status set to 503, if the timeout value is reached and no timeout handler is registered. The exception MUST be processed as described in section~\ref{method_exc}. If a registered timeout handler resets the timeout value or resumes the connection and returns a response, \jaxrs\ implementations MUST NOT generate an exception.

It is also possible to register callbacks on an instance of \code{AsyncResponse} in order to listen for processing completion (\code{CompletionCallback}) and connection termination (\code{ConnectionCallback}) events. See Javadoc for \code{AsyncResponse} for more information on how to register these callbacks. Note that support for \code{ConnectionCallback} is OPTIONAL.


\section{EJB Resource Classes}
\label{async_ejbs}

As stated in Section~\ref{ejbs}, JAX-RS implementations in products that
support EJB must also support the use of stateless and singleton session beans
as root resource classes. When an EJB method is annotated with \code{@Asynchronous}, the 
EJB container automatically allocates the necessary resources for its execution. 
Thus, in this scenario, the use of an \code{Executor} is unnecessary to generate
an asynchronous response.

Consider the following example:

\begin{listing}{1}
@Stateless 
@Path("/")
class EJBResource {

    @GET @Asynchronous
    public void longRunningOp(@Suspended AsyncResponse ar) {
        executeLongRunningOp();
        ar.resume("Hello async world!");
    }
}
\end{listing}

There is no explicit thread management needed in this case since that is
under the control of the EJB container. Just like the other examples in this chapter,
the response is produced by calling \code{resume} on the injected \code{AsyncResponse}. Hence, the return type of \code{longRunningOp} is simply \code{void}.

\section{Client API}
\label{client_api_async}

The fluent API supports asynchronous invocations as part of the invocation building process. By default, invocations are synchronous but can be set to run asynchronously by calling the \code{async} method and (optionally) registering an instance of \code{InvocationCallback} as shown next:

\begin{listing}{1}
Client client = ClientBuilder.newClient();
WebTarget target = client.target("http://example.org/customers/{id}");
target.resolveTemplate("id", 123).request().async().get(
    new InvocationCallback<Customer>() {
        @Override
        public void completed(Customer customer) {
            // Do something
        }
        @Override
        public void failed(Throwable throwable) {
            // Process error
        }
    });
\end{listing}

Note that in this example, the call to \code{get} after calling \code{async} returns immediately without blocking the caller's thread.
The response type is specified as a type parameter to \code{InvocationCallback}. The method \code{completed} is called when the invocation completes successfully and a response is available; the method \code{failed} is called with an instance of \code{Throwable} when the invocation fails.

All asynchronous invocations return an instance of \code{Future<T>} here the type parameter \code{T} matches the type specified in \code{InvocationCallback}. This instance can be used to monitor or cancel the asynchronous invocation:

\begin{listing}{1}
Future<Customer> ff = target.resolveTemplate("id", 123).request().async()
    .get(new InvocationCallback<Customer>() {
        @Override
        public void completed(Customer customer) {
            // Do something
        }
        @Override
        public void failed(Throwable throwable) {
            // Process error
        }
    });

// After waiting for a while ...
if (!ff.isDone()) {
    ff.cancel(true);
} 
\end{listing}

Even though it is recommended to pass an instance of \code{InvocationCallback} when executing an asynchronous call, it is not mandated. When omitted, the \code{Future<T>} returned by the invocation can be used to gain access to the response by calling the method \code{Future.get}, which will return an instance of \code{T} if the invocation was successful or \code{null} if the invocation failed.






